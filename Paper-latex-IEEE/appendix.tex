\section{Appendix}
\subsection{Testing for goodness-of-fit of a power law distribution}

The goodness-of-fit tests we employed were built following a procedure indicated by Clauset, Shalizi and Newman \cite[pp. 15-18]{clauset2009power}. What follows summarizes it in the context of the paper. The test's null hypothesis is that the empirical data are distributed according to a power law model; the alternative hypothesis is that they are not.

First, we fit the data for the degree distribution of a network generated by our model to a discrete power-law model, using maximum likelihood estimation. When we are testing for goodness-of-fit of the entire degree distribution, we set the fitted power-law model lower bound to 1; when we are testing for goodness-of-fit of the distribution's upper tail only, we choose a lower bound  such that the Kolmogorov-Smirnov distance $D$ between the power law model and the empirical data is minimized. Formally, define

\[D = \underset_{k>=k_{min}{max}} | S(k) - P(k)| \]

Here, $S(k)$ is the cumulative density function of the data for the observations with value at least $k_{min}$, and $P(k)$ is the cumulative density function for the power-law model that best fits the data in the region $k>=k_{min}$. The value of $k_{min}$ that minimizes the function $D$ is the estimate for the model's lower bound.

Next, we generate a large number of power-law distributed synthetic datasets with the same scaling parameter, standard deviation and lower bound as those of the distribution that best fits the empirical data. We fit each of these synthetic datasets to its own power-law model and calculate the $D$ statistics of each one relative to its own model. Finally, we count what fraction of the values of $D$ thus computed is larger than the value of $D$ computed for the empirical data. This fraction is interpretable as a p-value: the probability that data generated by our estimated best-fit power-law model will be more distant from the model than our empirical data (``distant'' in the Kolmogorov-Smirnov sense). A p-value close to zero indicates that it is quite unlikely that the estimated power-law model would generate empirical data so distant from the fitted power function; a p-value close to one, on the contrary, indicates that the estimated power model is quite likely to generate empirical data that are further away from the fitted power function than the ones we collected. 

Generating artificial datasets requires a treatment for the region below $k_{min}$  that differs from that of the one above it. We proceed as follows. Assume that our observed dataset has $n$ observations total and $n_{tail}$ observations such that $k>=k_{min}$. To generate a synthetic datasets with $n$ observations, we repeat the following procedure $n$ times:
\begin{itemize}
\item With probability $n/n_{tail}$ we generate a random number $k_i$ with $k_i>=k_{min}$, drawn from a power law with the same scaling parameter as our best-fit model.
\item Otherwise, with probability $1 - n/n_{tail}$, we select one element uniformly at random from among the elements of the observed dataset in the region $k<k_{min}$.
\end{itemize}

At the end of the process, we will have a synthetic dataset that follows the estimated power-law model for $k>=k_{min}$, but has the same non-power law distribution below $k_{min}$.

This test requires we decide how many synthetic datasets to generate for each test; and what is the threshold value below which we reject the null hypothesis. Again based on \cite{clauset2009power} we make the following decisions:

\begin{itemize}
\item We set the number of artificial datasets generated to 2500. This corresponds to an accuracy of about 0.01, based on an analysis of the expected worst-case performance of the test. 
\item We conservatively set the rejection threshold at 0.01.
\end{itemize}

\subsection*{A2. Choosing parameter values}

The computer simulation's computational intensity prevented us from conducting a thorough exploration of its behaviour across the whole parameter space. It follows we had to pick values from some parameters. In this section we discuss briefly our choice of parameter values.
The choice of $m=1$ implies that the number of edges in the networks in our control group will be equal to the number of nodes; we initialize the network with two nodes connected by two edges (one in each direction), then add one node and one edge at each time step. A glance at  Figure \ref{fig:NetViz} shows that this is unrealistic. The real-world online communities described in section 3.1 all display a number of edges with is a multiple of the number of nodes. We justify this choice as follows: in a real-world thriving online community there are many things going on at any given time. Both the spontaneous activities of its participants and the command-and-control work of online community managers and moderators follow multiple logics. 
 
In the paper, we are interested in pitting against each other two of these logics, that of preferential attachment, that tends to generate rich-gets-richer phenomena; and that of onboarding, that tends to introduce a measure of equality. The way we modeled onboarding is by having one single incoming edge targeting the only newcomer to the community at each timestep; we therefore chose to have one single non-onboarding generated edge at each timestep. It seems reasonable that our choice would make  these two forces roughly equivalent to each other, and make the impact of onboarding on the in-degree distribution easier to detect.

The choice of $A=1$ follows from another, and more fundamental, modeling choice. We mimic Dorogovtsev's and Mendes's approach, where the network being modeled is directed and the probability of a new edge to target a node with in-degree $k$ is proportional to $k$ \cite{dorogovtsev2002evolution}; this contrasts with Barab\'asi's and Albert's approach, that models the network as undirected and assumes that the probability of a new edge to target a node is proportional to its total degree. In a Dorogovtsev-Mendes type model, new nodes have, by construction, in-degree zero, whereas in a Barab\'asi-Albert type model new nodes have total degree one. It follows that, in a Dorogovtsev-Mendes type model, the parameter $A$ tunes the "traction"�� of preferential attachment: the higher its value, the weaker the grip of pure preferential attachment. For $A=0$ Dorogovtsev-Mendes type models degenerate into "multiple star networks", where the probability of newcomers to receive an edge is zero, and all edges target the nodes initially in the network for all time. 

Setting $A = 1$ we make the probability of a newcomer to receive its first edge equal to one half that of an incumbent participant who already has one incoming edge to receive its second one, one third of that of an incumbent participant who already has two incoming edges to receive its third one and so on. One can check that this behaviour mimics that of the simplest, and best known, Barab\'asi-Albert type model. 

\subsection*{A.3 Regression analyses}

Full results of the regression analyses mentioned in section \ref{sec:results} are provided below.

\begin{table}[h]
\centering
\caption{Regressing dummy variables representing values of onboarding effectiveness $\nu_1$ and community responsiveness $\nu_2$ on p-values of the goodness-of-fit tests. $\nu_1 = 0$ and $\nu_2 = 0$ act as the baseline}
\label{table:GOF2}
\begin{tabular}{|l|r|r|r|r|} 
\hline
  & Coef. & Std. Err. & t & $P>|t|$ \\
\hline  
$\nu_1$ & & & & \\
0.2        & .003592        & .0150601        & 0.24        & 0.811   \\
0.4          & -.007352        & .0150601        & 3.01        & 0.003    \\
0.6          & -.011376        & .0150601        &  2.46        &  0.014    \\
0.8          & .073304        & .0150601        & 4.87        &  0.000  \\
1.0          & .041556        & .0150601        & 2.76        & 0.006  \\
\hline  
$\nu_2$ & & & & \\
0.2        & .000784        & .0150601        &  0.05        & 0.958   \\
0.4          & -.007352        & .0150601        & -0.49        & 0.625    \\
0.6          & -.011376        & .0150601        &  -0.76        &  0.450  \\
0.8          & -.00424        & .0150601        & -0.28        &  0.778  \\
1.0          & -.007916        & .0150601        & -0.53        & 0.599  \\
\hline
$\nu_1$ * $\nu_2$ & & & & \\
0.2 * 0.2 & .015968 & .0212982 & 0.75 & 0.453 \\
0.2 * 0.4 & .029696 & .0212982 & 1.39 & 0.163 \\
0.2 * 0.6 & .031884 & .0212982 & 1.50 & 0.453 \\
0.2 * 0.8 &  .023488 & .0212982 & 1.10 & 0.270 \\
0.2 * 1.0 & 024608 & .0212982 & 1.16 & 0.248 \\
0.4 * 0.2 &  -.008472 & .0212982 & -0.40 & 0.691 \\
0.4 * 0.4 &  .001268 & .0212982 & 0-06 & 0.953 \\
0.4 * 0.6 & -.01024 & .0212982 & -0.48 & 0.631 \\
0.4 * 0.8 &  -.017532 & .0212982 & -0.82 & 0.410 \\
0.4 * 1.0 & .018888  & .0212982 & 0.89 & 0.375 \\
0.6 * 0.2 & -.011688  & .0212982 &   -0.55 &  0.583 \\
0.6 * 0.4 & .013072  & .0212982  &   0.61 &  0.539 \\
0.6 * 0.6 & .041848  & .0212982   &  1.96 &  0.050 \\
0.6 * 0.8 & -.001572  & .0212982  &  -0.07 &  0.941 \\
0.6 * 1.0 & -.008832  & .0212982  &  -0.41 &  0.678 \\
0.8 * 0.2 &  -.018224 &  .0212982  &  -0.86 &  0.392 \\
0.8 * 0.4 &  -.021648  & .0212982  &  -1.02 &  0.309 \\
0.8 * 0.6 & -.012132 &  .0212982   & -0.57&   0.569 \\
0.8 * 0.8 &  -.009548  & .0212982 &   -0.45 &  0.654 \\
0.8 * 1.0 & -.001912 &  .0212982  & -0.09 &  0.928 \\
1.0 * 0.2 &  .019024  & .0212982   &  0.89 &  0.372 \\
1.0 * 0.4 &  .039048  & .0212982  &   1.83  & 0.067 \\
1.0 * 0.6 & .026956  & .0212982  &   1.27 &  0.206 \\
1.0 * 0.8 & .026396 &  .0212982  &   1.24 &  0.215 \\
1.0 * 1.0 & .027512 &  .0212982  &   1.29  & 0.197 \\
\hline
constant & .059312  & .0106491  &   5.57 &  0.000 \\
\hline

\end{tabular}
\end{table}


\begin{table}[h]
\centering
\caption{Regressing dummy variables representing values of onboarding effectiveness $\nu_1$ and community responsiveness $\nu_2$ on $k_{min}$. $\nu_1 = 0$ and $\nu_2 = 0$ act as the baseline}
\label{table:regression kmin}
\begin{tabular}{|l|r|r|r|r|} 
\hline
  & Coef. & Std. Err. & t & $P>|t|$ \\
\hline  
$\nu_1$ & & & & \\
0.2        & .07 &  .1812853   &  0.39 &  0.699   \\
0.4          & .32  & .1812853  &   1.77 &  0.078   \\
0.6          & .43  & .1812853   &  2.37  & 0.018     \\
0.8          & .35  & .1812853   &  1.93  & 0.054   \\
1.0          & .28  & .1812853  &   1.54 &  0.123  \\
\hline  
$\nu_2$ & & & & \\
0.2        & .03  & .1812853  &   0.17  & 0.869    \\
0.4          & -.01   &.1812853  &  -0.06 &  0.956    \\
0.6          &  -.14  & .1812853  &  -0.77  & 0.440  \\
0.8          &  -.01 &  .1812853  &  -0.06 &  0.956  \\
1.0          &   .1  & .1812853  &   0.55 &  0.581  \\
\hline
$\nu_1$ * $\nu_2$ & & & & \\
0.2 * 0.2 &  -.06 &  .2563762 &   -0.23 &  0.815 \\
0.2 * 0.4 & .12  & .2563762  &   0.47 &  0.640 \\
0.2 * 0.6 & .13  & .2563762   &  0.51 &  0.612 \\
0.2 * 0.8 &  -.1  & .2563762  &  -0.39  & 0.697 \\
0.2 * 1.0 &  -.06  & .2563762   & -0.23  & 0.815 \\
0.4 * 0.2 &  -.12  & .2563762   & -0.47 &  0.640 \\
0.4 * 0.4 &   -.05  & .2563762  &  -0.20 &  0.845 \\
0.4 * 0.6 & -.15  & .2563762  &  -0.59 &  0.559 \\
0.4 * 0.8 &  -.15  & .2563762  &  -0.59 &  0.559 \\
0.4 * 1.0 & -.09  & .2563762  &  -0.35 &  0.726 \\
0.6 * 0.2 & -.17  & .2563762  &  -0.66  & 0.507 \\
0.6 * 0.4 &  -.16 &  .2563762  &  -0.62 &  0.533 \\
0.6 * 0.6 &  .05  & .2563762  &  0.20 &  0.845 \\
0.6 * 0.8 & -.1  & .2563762  &  -0.39  & 0.697 \\
0.6 * 1.0 & -.45  & .2563762 &   -1.76 &  0.079 \\
0.8 * 0.2 &  .29 &  .2563762  &   1.13 &  0.258 \\
0.8 * 0.4 &   1.23e-13 &  .2563762   &  0.00 &  1.000 \\
0.8 * 0.6 & .05  & .2563762  &  0.20 &  0.845 \\
0.8 * 0.8 &   -.05  & .2563762  &  -0.20 &  0.845 \\
0.8 * 1.0 & -.06 &  .2563762 &   -0.23 &  0.815 \\
1.0 * 0.2 &   .39  & .2563762  &   1.52 &  0.128 \\
1.0 * 0.4 &  .34  & .2563762   &  1.33 &  0.185 \\
1.0 * 0.6 & .33  & .2563762  &   1.29  & 0.198  \\
1.0 * 0.8 & .24 &  .2563762   &  0.94  & 0.349 \\
1.0 * 1.0 & -.02 &  .2563762  &  -0.08 &  0.938 \\
\hline
constant &  3.88  & .1281881  &  30.27 &  0.000 \\
\hline

\end{tabular}
\end{table}

\begin{table}[h]
\centering
\caption{Regressing dummy variables representing values of onboarding effectiveness $\nu_1$ and community responsiveness $\nu_2$ on the power-law model's exponent for $k>= 1$. $\nu_1 = 0$ and $\nu_2 = 0$ act as the baseline}
\label{table:regression exponent-all}
\begin{tabular}{|l|r|r|r|r|} 
\hline
  & Coef. & Std. Err. & t & $P>|t|$ \\
\hline  
$\nu_1$ & & & & \\
0.2        & -.0392798  & .0005885 &  -66.75 &  0.000   \\
0.4          &  -.0698609  & .0005885 & -118.71 &  0.000  \\
0.6          & .-.0955784 &  .0005885 & -162.41 &  0.000     \\
0.8          & -.1163595  & .0005885 & -197.72 &  0.000   \\
1.0          & -.1348692  & .0005885 & -229.17  & 0.000  \\
\hline  
$\nu_2$ & & & & \\
0.2        & .0002897  & .0005885  &   0.49  & 0.623   \\
0.4          &  -.0002099  & .0005885  &  -0.36  & 0.721    \\
0.6          &   -.000763  & .0005885 &   -1.30  & 0.195  \\
0.8          &   -.0007079  & .0005885 &   -1.20 &  0.229  \\
1.0          &    -.0002022  & .0005885  &  -0.34  & 0.731   \\
\hline
$\nu_1$ * $\nu_2$ & & & & \\
0.2 * 0.2 &   -.0002119 &  .0008323  &  -0.25 &  0.799 \\
0.2 * 0.4 & .0000679 &  .0008323  &   0.08  & 0.935  \\
0.2 * 0.6 & .00126 &  .0008323   &  1.51 &  0.130  \\
0.2 * 0.8 & .0009963  & .0008323   &  1.20 &  0.231 \\
0.2 * 1.0 &  .0003897  & .0008323  &   0.47  & 0.640   \\
0.4 * 0.2 &  -.0001497 &  .0008323  &  -0.18  & 0.857  \\
0.4 * 0.4 &   .0006467  & .0008323  &   0.78 &  0.437 \\
0.4 * 0.6 &  .0002466 &  .0008323  &   0.30 &  0.767 \\
0.4 * 0.8 &   .0001724 &  .0008323   &  0.21 &  0.836 \\
0.4 * 1.0 &  .0010392 &  .0008323   &  1.25 &  0.212  \\
0.6 * 0.2 & -.0001797  & .0008323 &   -0.22 &  0.829 \\
0.6 * 0.4 &   .0003026 &  .0008323  &   0.36  & 0.716 \\
0.6 * 0.6 &  .0001388  & .0008323   &  0.17 &  0.868\\
0.6 * 0.8 & .0006951  & .0008323  &   0.84  & 0.404  \\
0.6 * 1.0 &  .0002917  & .0008323   &  0.35  & 0.726  \\
0.8 * 0.2 &   -.0009096  & .0008323  &  -1.09 &  0.275  \\
0.8 * 0.4 &   .0002595 &  .0008323  &   0.31  & 0.755 \\
0.8 * 0.6 & .0002607  & .0008323  &   0.31 &  0.754 \\
0.8 * 0.8 &   .0007477 &  .0008323  &   0.90 &  0.369 \\
0.8 * 1.0 & .0008467 &  .0008323    & 1.02  & 0.309  \\
1.0 * 0.2 &    -.0002435 &  .0008323  &  -0.29  & 0.770 \\
1.0 * 0.4 &  3.13e-06  & .0008323  &   0.00 &  0.997 \\
1.0 * 0.6 & -.0000185  & .0008323   & -0.02 &  0.982  \\
1.0 * 0.8 &  .0005968  & .0008323  &   0.72 &  0.473  \\
1.0 * 1.0 & .0002941  & .0008323  &   0.35  & 0.724 \\
\hline
constant &  1.887763  & .0004161&  4536.42 & 0.000 \\
\hline

\end{tabular}
\end{table}

\begin{table}[h]
\centering
\caption{Regressing dummy variables representing values of onboarding effectiveness $\nu_1$ and community responsiveness $\nu_2$ on the power-law model's exponent for $k>k_{min}$. $\nu_1 = 0$ and $\nu_2 = 0$ act as the baseline}
\label{table:regression exponent}
\begin{tabular}{|l|r|r|r|r|} 
\hline
  & Coef. & Std. Err. & t & $P>|t|$ \\
\hline  
$\nu_1$ & & & & \\
0.2        &  -.1305077 &  .0256608  &  -5.09 &  0.000   \\
0.4          &  -.2393938  & .0256608  &  -9.33 &  0.000  \\
0.6          & .-.0955784 &  .0005885 & -162.41 &  0.000     \\
0.8          & -.4230415  & .0256608   & -16.49 &  0.000   \\
1.0          & -.4895831  & .0256608  & -19.08 &  0.000  \\
\hline  
$\nu_2$ & & & & \\
0.2        & .0034937 &  .0256608   &  0.14  & 0.892   \\
0.4          &  .0147113  & .0256608   &  0.57 &  0.566     \\
0.6          &   -.0231297 &  .0256608  &  -0.90 &  0.367   \\
0.8          &  .0185004  & .0256608   &  0.72 &  0.471  \\
1.0          &    .0298063  & .0256608    & 1.16 &  0.245   \\
\hline
$\nu_1$ * $\nu_2$ & & & & \\
0.2 * 0.2 &   -.006399  & .0362899   & -0.18 &  0.860 \\
0.2 * 0.4 &  -.0015971  & .0362899  &  -0.04 &  0.965   \\
0.2 * 0.6 &  .0023115 &  .0362899   &  0.06 &  0.949  \\
0.2 * 0.8 & -.0525311 &  .0362899   & -1.45 &  0.148  \\
0.2 * 1.0 &   -.0307629  & .0362899  &  -0.85 &  0.397    \\
0.4 * 0.2 &   -.0226558 &  .0362899  &  -0.62 &  0.532   \\
0.4 * 0.4 &   -.0253722  & .0362899  &  -0.70 &  0.485 \\
0.4 * 0.6 &   .0020922 &  .0362899  &   0.06  & 0.954 \\
0.4 * 0.8 &   -.0252102 &  .0362899 &   -0.69 &  0.487  \\
0.4 * 1.0 &   -.0270287 &  .0362899  &  -0.74  & 0.456   \\
0.6 * 0.2 & -.0096661  & .0362899  &  -0.27  & 0.790  \\
0.6 * 0.4 &   -.0443246 &  .0362899  &  -1.22  & 0.222  \\
0.6 * 0.6 &  .0113786  & .0362899   &  0.31  & 0.754\\
0.6 * 0.8 &  -.0234321  & .0362899 &   -0.65  & 0.519   \\
0.6 * 1.0 &   -.0680064 &  .0362899  &  -1.87  & 0.061   \\
0.8 * 0.2 &    .0359783  & .0362899   &  0.99 &  0.322  \\
0.8 * 0.4 &    -.00544 &  .0362899  &  -0.15 &  0.881 \\
0.8 * 0.6 &  .0144717 &  .0362899  &   0.40  & 0.690  \\
0.8 * 0.8 &    -.026186  & .0362899 &   -0.72 &  0.471\\
0.8 * 1.0 &  -.0388437 &  .0362899  &  -1.07 &  0.285   \\
1.0 * 0.2 &     .0283216  & .0362899  &   0.78 &  0.435 \\
1.0 * 0.4 &   .007842 &  .0362899  &   0.22  & 0.829  \\
1.0 * 0.6 &  .0417675  & .0362899  &   1.15 &  0.250   \\
1.0 * 0.8 &  .0001551  & .0362899  &   0.00 &  0.997   \\
1.0 * 1.0 &  -.0268194 &  .0362899 &   -0.74 &  0.460  \\
\hline
constant &   2.985189 & .0181449 &  164.52 &  0.000 \\
\hline

\end{tabular}
\end{table}


\subsection*{A.4 Software stack}

To obtain the results discussed in the paper we made use of the following software applications:
\begin{itemize}
\item Graphs were handled through the networkx Python library.
\item Parallel computation was handled through the multiprocessing Python package. 
\item The estimation of power-law models and the generation of synthetic data made use of the PowerLaw Python package developed by Alstott et al. \cite{alstott2014powerlaw}.
\item The network visualizations were drawn with Edgesense: https://github.com/Wikitalia/edgesense
\end{itemize}
