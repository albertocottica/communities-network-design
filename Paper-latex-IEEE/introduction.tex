\section{Introduction}

Online communities are used to aggregate and process information dispersed across many individuals. Pioneered in the 1980s, they have become more widespread with mass adoption of the Internet, and are now used across many different contexts in business \cite{mcwilliam2012building, tapscott2008wikinomics}, politics and public decision making \cite{rheingold1993virtual, noveck2009wiki, cottica2010wikicrazia}, expertise sharing \cite{rheingold1993virtual, zhang2007expertise, shirky2008here}, and education \cite{milligan2013patterns}. Most online communities lack a central command structure; despite this, many display remarkably coherent behaviour, and have proven effective at large tasks like writing the largest encyclopedia in human history (Wikipedia), providing an always-on free helpline for software engineering problems (StackOverflow), or building a detailed map of planet Earth (OpenStreetMap) \cite{shirky2008here}. 

Organizations running online communities typically employ community managers, tasked with encouraging participation and resolving conflict: this practice is almost as old as online communities themselves and predates the Internet \cite{rheingold1993virtual}, although it has become much more widespread as Internet access became a mass phenomenon. Though most participants to online communities are unpaid and answer to no one, a small number of them (only one or two in the smaller communities, many more in the larger ones) will recognize some central command, and carry out its directives. We shall henceforth call such directives \emph{policies}. 

Putting in place management policies for online communities is costly. Professional community managers need to be recruited, trained and paid; software tools to monitor communities and make their work possible need to be developed and maintained. This raises the question of why organisations running online communities choose certain policies, and not others. A full investigation of this matter is outside the scope of this paper; however, in what follows we outline and briefly discuss the set of assumptions that underpin our investigation. 

\begin{enumerate}
\item In line with the network science approach to online communities, we model online communities as social networks of interactions across participants. 
\item We assume that organisations can be modelled as economic agents maximizing some objective function. The target variable being maximized can be profit (for online communities run by commercial companies); or welfare (for online communities run by governments or other nonprofit entities); or some combination of the two. 
\item We assume that the topology of the interaction network characteristic of online communities affects their ability to contribute to the maximisation of the target variable. Indications that this assumpion might be reasonable are not difficult to find in the literature. Some examples:
\begin{itemize}
	\item When IBM decided to contribute to the development of the open source operating system Linux, it decided to unplug the project team from IBM's corporate communication network, and instructed them to adopt the communication tools of the Linux community instead. This radical reshaping of IBM's interaction network is reported to have had a great impact on the productivity of IBM programmers involved in that effort [38]. 
	\item American IT service company Geek Squad shelved their elaborate in-house design collaboration platform as their staff self-organized on an existing, third-party virtual hangout space \cite{tapscott2008wikinomics}. 
	\item Mainstream social networks like Facebook are constantly �\emph{rewiring} the interaction network across their users to ensure ever more of of them watch ever more, better targeted and more effective ads, therefore enhancing their revenue \cite{slegg2014facebook}. 
	\end{itemize}
\item We assume that such organisations choose their policies as follows: 
\begin{itemize} 
	\item Solve their maximisation problem over network topology. This yields a vector of desired network characteristics, where “desired” means that those characteristics define a maximum of the objective function. These solutions will have the form “In order to best meet our ultimate [profit or welfare] goals, the interaction network in our online community should be in state formula“, where formula is a vector of topology-related parameters.
	\item Derive a course of action that community managers could take to change the network away from its present state formulato the desired state formula.
	\item Encode such course of action in a set of simple instructions for community managers to execute. Computer scientists might think of such instructions as algorithms; economists call them mechanisms; professional online community managers call them policies. In this paper we use this third term. 
\end{itemize}
\end{enumerate}

All this implies that the decision to deploy a particular policy on an online community is very sophisticated indeed. It requires an understanding of how the shape of the interaction network within the community affects the organization's ultimate goals; therefore, to a first approximation, it can be understood as a network design exercise. And yet, interaction networks in online communities cannot really be designed; they are the result of many independent decisions, made by individuals who do not respond to the organisation's command structure. Interaction networks are, to a large extent, emergent. We conclude that an online community management policy is best understood as an attempt to “lead”, “influence”, “accompany”, “nudge” emergent social dynamics (these words are all  frequent among online community management professionals); to use a more synthetic expression, it can be best understood as the attempt to design for emergence. Its paradoxical nature is at the heart of its appeal. 

In the paper, we do not attempt to model the whole chain of decision starting from the maximization problem at step 2. Our work starts where that chain ends: we are interested in detecting the mathematical signature of one particular policy, onboarding, in the network topology. Our approach, however, does rest on the idea that organisations running online communities are trying to superimpose an element of design onto the emergent social dynamics characterizing them; that they are doing this according to a plan, which involved solving an optimisation problem; and that such plan is formulated in terms of a desired network shape. 

We consider a policy called \emph{onboarding}, perhaps one of the simplest and most common in online community management \cite{rheingold1993virtual, shirky2008here}. As a new participant becomes active (for example by posting her first post, or commenting somebody else’s post for the first time), professional community managers are instructed to leave her a comment that contains (a) friendly, positive feedback and (b) suggestions to engage with other, existing participants that she might have interests in common with. A new participants, after her first contribution, would get a comment like this by one of the moderators:

\emph{``Welcome, Alice! That was a very interesting point. It definitely resonates with my own experience in the field. In our community, the people who are most involved in the matter are Bob [link] and Charlie [link]. You might be interested in this post [link] by Bob, where he relates his own experience: if you leave him a comment, I am sure an interesting conversation will ensue.''}

If the new participant engages (by replying or asking a question), she gets another message (an acknowledgement or an answer); after that, the new participant is generally considered onboarded, and is not made the object of any special attention.

This paper considers the issue of why organizations running online communities invest considerable resources in onboarding, and how can they determine whether their investment has been productive. We do so by modeling online conversations as social networks of interactions across participants, and looking for the effect that onboarding has on the topology of those networks.

Section 2 briefly examines the two strands of literature that we mostly draw upon. Section 3 presents some data from real-world online communities; then proceeds to describe our main experiment, based on a computer simulation of interaction in online communities with and without onboarding. Section 4 presents the experiment's results. Section 5 discusses them.
