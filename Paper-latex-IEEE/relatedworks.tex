\section{Literature survey}

The extraordinary successes of online communities in deploying large-scale, decentralized projects has led many scholars to conjecture that online communities exhibit emergent behavior, and called such behavior collective intelligence, after an influential book by Pierre Lévy \cite{pierre1997collective}. This name was adopted by a research community that aims at providing tools for better collective sense- and decision making such as argument maps (representations of the logical structure of a debate, with all redundancy eliminated) \cite{shum2003roots} and attention-mediation metrics (indicators that signal what, in an online debate, is worthiest reading and responding to. The number of Likes on Facebook is one such metric) \cite{klein2012enabling}. 

Collective intelligence scholars acknowledge the existence and importance of online community management practices – indeed, they have tried for some time to systematize it \cite{blondel2008fast, diplaris2011emerging} and produce technological innovation to support it \cite{de2012contested}. An important part of their research program aims to better equip community managers to carry out this work; for example, argument maps are advocated on the basis that the effort to build them will engage more users in the debate, and will do so more effectively than competing engagement techniques \cite{shum2003roots}. These tools are meant to facilitate and encourage participation to online communities, to make it easier for individuals to extract knowledge from them.

Starting in the 2000s, online communities became the object of another line of enquiry, stemming from network science. Network representation of relationships across groups of humans has yielded considerable insights in social sciences since the work of the sociometrists in the 1930s, and continues to do so; phenomena like effective spread of information, innovation adoption, and brokerage have all been addressed in a network perspective \cite{borgatti2009network, burt2009structural}. As new datasets encoding human interaction became available, many online communities came to be represented as social networks. This was the case for social networking sites, like Facebook \cite{lewis2008tastes, nick2013toward}; microblogging platform like Twitter \cite{kunegis2013preferential, java2007we, hodas2014simple}; news-sharing services like Digg \cite{hodas2014simple}; collaborative editing projects like Wikipedia \cite{laniado2011wikipedians}; discussion forums like the Java forum \cite{zhang2007expertise}; and bug reporting services for software developers like Bugzilla \cite{zanetti2012quantitative}). Generally, such networks represent participants as nodes. Edges represent a relationship or interaction which varies according to the nature of the online community in question: friendship for Facebook; follower-followed relationship, retweet or mention in Twitter; vote or comment in Digg and the Java forum; talk in Wikipedia; comment in Bugzilla. 

In contrast to collective intelligence scholars, network scientists typically do not address the issue of community management, and treat social networks drawn from online interaction as fully emergent. In this paper, we employ a network approach to investigate the issue of whether the work of community managers leaves a footprint detectable by quantitative analysis, and of what kind. By doing so, we seek to make community management practices easier and more accountable to monitor, with a view of making online communities better at achieving their goals, be them maintaining an encyclopedia or collectively writing a new constitution. 

In particular, we exploit a result from the theory of evolving networks. This substantial branch of the literature on networks originates from seminal work by Barabási and Albert \cite{barabasi1999emergence} in 1999. Departing from previous network theory work, they observe that most networks in nature grow over time, with new nodes adding themselves to the network; and that new nodes do normally not connect to existing nodes with equal probability, but rather prefer to link to nodes that are already highly connected. They then show that the assumption of growth and preferential attachment, when taken together, result in a network whose degree distribution converges to a power law: 

This result holds for any network that displays growth and preferential attachment, and the exponent is shown to be analytically equal to 3 \cite{barabasi2005origin, barabasi1999mean}.  The model was later generalized by Dorogovtsev and Mendes to allow for non-preferential attachment to coexist with preferential attachment; new edges forming between existing nodes; and for some nodes to be more attractive than others. The generalized model was shown to converge to a power law with exponent ranging from 2 to infinity. This prediction was confirmed across a broad range of networks, including many social networks \cite{dorogovtsev2002evolution}.

With a view to this goal, we propose that Dorogovtsev’s and Mendes’s result \cite{dorogovtsev2002evolution} holds for interaction in online communities, too. More explicitly, we assume that the network representing interaction in an online community, left to its own device – in the absence of community management policies, will converge to a degree distribution that follows a power law with exponent 2 or larger. We then use this prediction as a baseline state. A policy successfully enacted on the online community, with community managers being instructed to execute certain tasks, will result in its degree distribution deviating from the baseline power law in predictable ways. Such deviation can be interpreted as the signature that the policy is working well. 

The most important one difficulty in this method is the absence of a counterfactual: if a policy is enacted in the online community, the baseline degree distribution corresponding to the absence of the policy is not observable, and viceversa. This rules out a direct proof that the policy “works”. Instead, we proceed as follows: 

\begin{enumerate}
\item We initially examine data from three small online communities. Two of them deploy an active policy to welcome new members and integrate them with the incumbent ones (a practice sometimes called onboarding), while the third one does not. We observe that, indeed, the shape of the degree distribution of the first two differs from that of the third.  
\item We propose an experiment protocol to determine whether onboarding policies can explain the differences observed between the degree distributions of the first two online communities and that of the third one. 
\item We simulate the growth of online communities by means of a computer model. Variants to the model cover the relevant cases: the absence of onboarding policies and their presence, with varying degrees of effectiveness. 
\item We run the experiment protocol against the degree distributions generated by the computer model, and discuss its results.
\end{enumerate}