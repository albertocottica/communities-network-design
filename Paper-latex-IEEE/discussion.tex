\section{Discussion}

\subsection{Accounting for degree distribution shape in the interaction networks of online communities}

Our simulation model incorporates two forces. The first one is preferential attachment; the second is onboarding. The former is meant to represent the rich-get-richer effect observed in many real-world social networks; the latter is meant to represent the onboarding action of moderators and community managers. The former's effect is known to lead to the emergence of an in-degree distribution that approximates a power-law model. The latter's effect is more subtle, because it is in turn composed of two effects. The first one consists in the direct action of the moderator, which  always targets the newcomer; the second one by the actions that might be undertaken as a result of well-executed onboarding policy. 

The direct action of the moderators creates edges pointing to nodes not selected by preferential attachment – this is definitional of online community management. What (non-moderator) participants in the online community do as a result of moderator activity is not as clear cut. In our simulation model, fully successful onboarding results in extra edges, some of which point to nodes selected by preferential attachment, others to nodes selected otherwise. 

Also, onboarding only targets newcomers in online communities. As most of online community management policies, it concerns weakly connected participants in the community: moderators have no need to engage with very active, strongly connected participants, who clearly need no help in getting a conversation going. By doing so, moderators hope to help some shy newcomers turn into active, respected, sought out community members. Once this process is under way, however, moderators have no reason to continue to engage with the same individuals. In terms of our model, this means that newcomers, after having being onboarded, are going to receive new edges by preferential attachment only. It is therefore reasonable to expect that the degree distributions generated by our model display a heavy tail, with the frequency of highly connected nodes following a reasonable approximation of a power law. The overall result of onboarding, then, is an in-degree distribution with power-law behavior for high values of formula and non-power law behavior for low (close to 1) values of formula. This is indeed what we observe. 

Non-preferential attachment selection of edge targets leads to a poorer fit of power-law models to the in-degree distributions of the interaction networks of online communities where onboarding is present. This effect takes two forms. The first one is that, when onboarding is present, fitting a power-law model to the network's in-degree distribution and then running goodness-of-fit tests return a lower p-value than the p-value returned by the same test when onboarding is absent. The second effect is that, with onboarding, the value of formulathat minimizes the Kolmogorov-Smirnov distance between the best-fit power-law model and the observed data tends to be higher than without onboarding.

Our specification of the model accounts for an apparent paradox: the deviation of the observed networks' degree distributions from power-law behaviour is greater when onboarding is present, but completely ineffective. Ineffective onboarding only adds edges directly created by moderators, none of which are allocated across existing nodes by preferential attachment. As onboarding gets more effective, even more edges are added; some are allocated by preferential attachment, and drive the degree distribution back towards a pure power-law behavior. 

\subsection{Limitations}

We undertook this research work in the hope of discovering a simple test that could be used to assess the presence and effectiveness of online community management policies, onboarding among them. The guiding idea is that the agency of online community managers and moderators is guided by a logic other than the rich-get-richer dynamics that spontaneously arises in many social networks. Such dynamics is associated to power-law shaped degree distributions, which we can regard as the default state for social interaction networks: we conjecture that, whenever the interaction network of an online community does not have the shape of a power law, some agency is at work. We furthermore conjecture that the precise nature of such deviations can be interpreted, and ultimately that different social forces at work on an online community each leave their different mathematical signatures on the community's interaction network. Such signatures could, in principle, be read in the precise form of the deviation of the degree distribution from its default state. 

Our results are in accordance with the first conjecture. Applying onboarding to our simulated community, in whatever form, results in degree distributions that are markedly more distant from pure power-law behavior than we find in our control group. However, we do not find a monotonic relationship between onboarding's effectiveness and the distance of the resulting degree distribution from a pure power-law form. When we observe real-world online community whose interaction networks do not have a full power-law form (like those of Section 3.1); even though we are informed that they do employ onboarding policies; and even if we knew that onboarding policies are the only factor likely to affect the patterns of interaction within those communities, we could still not conclude from the form of their degree distributions how effective these policies are. 

The high dimensionality of the problem space is also a limitation. As stylized as it is, our model has several parameters interacting in complex ways: the number of edges generated at each time step, the additional attractiveness parameter, the probability that the newcomer reacts to the onboarding action, and the probability that such reaction attracts the attention of some other participant, that engages in conversation with the newcomer. This problem is compounded by the model's nontrivial computational intensity. 

\subsection{Directions for future research}

There are three obvious directions in which our work could be taken further. The most obvious one is a full and systematic exploration of the parameter space, with the goal of assessing our results' robustness with respect to model specification. In this paper we restrict ourselves to the presence and effectiveness of the onboarding action in a baseline model which is closely modeled on Dorogovtsev's and Mendes's results \cite{dorogovtsev2002evolution}; it would be useful to test for how these results carry through as we alter other parameters of the model, such as the number of edges  created at each time step, the additional attractiveness parameter . In the same vein, we would gain analytical granularity by resolving the effectiveness parameter  into its components  and .

A second direction for further research would be to attempt to make the model into a more realistic description of a real-world online community. Such an attempt would draw attention onto how some real-world phenomena, when incorporated in the model, influence its results. It would also carry the advantage of allowing online community management professional to more easily interact with the model and critique it. Several issues that could be investigated in this vein come to mind. For example, we could relax the assumption that the additional attractiveness parameter formula is identical for all nodes, allowing for different nodes in the network to attract incoming edges at different rates (a phenomenon known as multiscaling \cite{bianconi2001competition}). Secondly, we could introduce a relationship between out-degree and in-degree: this would reflect the fact that , in an online community, reaching out to others (which translates in increasing one's own out-degree in the interaction network) is a good way to get noticed and attract incoming comments (which translates in an increas in one's in-degree). Thirdly, we could work with online community manager professionals to model more precisely the action of online community managers: in this paper we assume that community managers recommend newcomers to reach out to existing participants, targeting with higher probability those with higher in-degrees; alternatives could be considered, for example uniformly random allocation or, in a model with multiscaling, targeting with higher probability participants with higher attractiveness. Finally, we could extend the model to consider not only onboarding, but other community management policies as well. Again, this would require close cooperation with online community management professionals.

A third direction for further research would attempt to gauge the influence of onboarding and other community management policies on network topology by indicators other than the shape of its degree distribution, such as the presence of subcommunities. 

